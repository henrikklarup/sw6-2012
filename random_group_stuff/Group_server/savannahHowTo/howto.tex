\documentclass[12pt]{report}
\usepackage{graphicx,epstopdf}
\usepackage[absolute]{textpos}
\usepackage[english]{babel}
\usepackage[latin1]{inputenc}
\usepackage{amsmath,amsfonts}
\usepackage{natbib}
\usepackage{fancyhdr}
\usepackage{wrapfig}
\usepackage{float}
\usepackage[Conny]{fncychap}
\usepackage[usenames,dvipsnames]{color}
\usepackage{longtable}
\usepackage{multirow}
\usepackage{algorithmic}
\setcitestyle{numbers,open={[},close={]}}
\usepackage{epsfig}
\usepackage[official,right]{eurosym}
\usepackage{rotating}
\usepackage{hyperref}
\usepackage{rotating}
\hypersetup{pdfborder={0 0 0}}
\definecolor{light-gray}{gray}{0.95}
\usepackage[absolute]{textpos}
\usepackage[rounded]{syntax}
\usepackage{appendix}
\grammarparsep 1pt
\usepackage{xyling}
\usepackage{subfigure}
\usepackage{slashbox}
\usepackage{verbatim}
\usepackage{float}

\newfloat{Code}{H}{myc}
\allowdisplaybreaks

%EPS images snask

%\usepackage{epstopdf}
\newif\ifpdf
\ifx\pdfoutput\undefined
   \pdffalse
\else
   \pdfoutput=1
   \pdftrue
\fi
\ifpdf
   \usepackage{graphicx}
   \usepackage{epstopdf}
   \DeclareGraphicsRule{.eps}{pdf}{.pdf}{`epstopdf #1}
   \pdfcompresslevel=9
\else
   \usepackage{graphicx}
\fi
%eps image snask end
\epstopdfsetup{suffix=}

%semantic udtryk
\usepackage{turnstile}
%$\nststile{Bottom}{Top}$

%\usepackage[tt]{titlepic}


%LOL MARTIN!
%End lool martin

% C# lol?
\usepackage{listings}
% default words comes from lstlang1.sty
\lstset{language=Java,
  basicstyle=\ttfamily\footnotesize\bfseries,
  float,
  columns=flexible,
  morekeywords=[1]{TmdbAPI,TmdbMovie},
  %keywordstyle=[1]\sffamily,
  backgroundcolor=\color{light-gray},
  captionpos=b,
  frame=single,
  breaklines=true, 
  keywordstyle=\color{Blue},
  commentstyle=\color{Green},
  stringstyle=\color{Mahogany},
  showspaces=false,
  showstringspaces=false,
  numbers=left,                   % where to put the line-numbers
  numberstyle=\footnotesize,      % the size of the fonts that are used for the line-numbers
  stepnumber=1
  }
  \newenvironment{program}

%End c-sharp lol

\usepackage{url}

%% Define a new 'leo' style for the package that will use a smaller font.
\makeatletter
\def\url@leostyle{%
  \@ifundefined{selectfont}{\def\UrlFont{\sf}}{\def\UrlFont{\small\ttfamily}}}
\makeatother
%% Now actually use the newly defined style.
\urlstyle{leo}


\pagestyle{fancy}
\lhead{}

\newcommand{\code}[1]{\texttt{#1}}
\newcommand{\secref}{section \ref}
\newcommand{\appref}{appendix \ref}
\newcommand{\chapref}{chapter \ref}
\newcommand{\figref}{figure \ref}
\newcommand{\tabelref}{table \ref}
\newcommand{\listref}{listing \ref}
\renewcommand{\headrulewidth}{0.4pt}
\renewcommand{\footrulewidth}{0.4pt}

%Rasmus' kind of lol
\makeatletter
\newenvironment{Figure}{%
\par\addvspace{12pt plus2pt}%
\def\@captype{figure}%
}{%
\par\addvspace{12pt plus2pt}%
}%
\long\def\@makecaption#1#2{%
\vskip\abovecaptionskip
\sbox\@tempboxa{#1: #2}%
\ifdim \wd\@tempboxa >\hsize
#1: #2\par
\else
\global \@minipagefalse
\hb@xt@\hsize{\hfil\box\@tempboxa\hfil}%
\fi
\vskip\belowcaptionskip}
\makeatother
% Rasmus' kind of lol - stop

\setlength{\headheight}{15pt}

%titlepage image halløj
%\usepackage{eso-pic}
%\newcommand\BackgroundPic{
%\put(0,0){
%\parbox[b][\paperheight]{\paperwidth}{%
%\vfill
%\centering
%\includegraphics[width=\paperwidth,height=\paperheight,keepaspectratio]{Images/front-page.png}%
%\vfill
%}}}
%halløj end


%Jesper Stuff
\lstset{
	language=SQL,
  breaklines=true,                                     % line wrapping on
  frame=ltrb,
  framesep=5pt,
  basicstyle=\normalsize,
  keywordstyle=\ttfamily\color{OliveGreen},
  identifierstyle=\ttfamily\color{CadetBlue}\bfseries,
  commentstyle=\color{Brown},
  stringstyle=\ttfamily,
  showstringspaces=ture
} %Skal nok lige ændre den her preamble hvis filen skal bruges i et andet projekt;-)

\title{Savannah how to}
\begin{document}

\maketitle
This is a short guide for using the savannah server project, while savannah should run on any OS that supports java, this guide will assume 
ubuntu server 11.04. Savannah has also been briefly tested on a windows machine, a few changes is required in the source code, more on this in the windows section.
\part*{Dependencies}
\label{depends}

\b{JDOM 1.1.3} \newline
\url{http://www.jdom.org/dist/binary/archive/jdom-1.1.3.tar.gz} \newline \\
\b{Connector/J mysql JDBC driver} \newline
\url{http://www.mysql.com/downloads/connector/j/}\newline \\
\b{JUnit} (only needed if you downloaded the source and want to avoid compiler warnings from junit test cases in the project \newline
\url{http://www.junit.org/} \newline \\
\b{Tomcat 6} (can be installed from a terminal in ubuntu 12.04) \newline
\url{http://tomcat.apache.org/} \newline


\part*{Ubuntu server}
the server needs java, mysql-server and tomcat installed, you can choose to install mysql and java during installation of the OS or install the packages afterwards.
This can be done by entering the following commands in a terminal
\begin{verbatim}
sudo apt-get update
sudo apt-get install mysql-server
sudo apt-get install openjdk-6-jre
\end{verbatim}

You can choose a different version of java if you like, however we have tested with openJDK-6, it should work just fine with oracle java(6 or 7) or openJDK-7, no guarantees though.

\part*{Installation of externals}
It is easiest to install the external libraries(JDOM,JUnit and mysql JDBC) directly on the systems jvm, do this by copying all .jar files to 
\begin{verbatim}
 /usr/lib/jvm/java-6-openjdk/jre/lib/ext
\end{verbatim}
If you use a different version of java, replace ``java-6-openjdk'' with your version.\newline\newline

if you are using ubuntu server 12.04, tomcat6 can be installed by typing 
\begin{verbatim}
 sudo apt-get install tomcat6
\end{verbatim}
in a terminal, if using any earlier version, like 11.04 you will need to download it manually from tomcats homepage, provided in the dependencies list.
and follow the installation guide on their homepage.


\part*{Source}
You can check out the source code from our Google code repository, use your favorite svn tool(tortoise on windows, or svn checkout on most Linux distroes)
\begin{verbatim}
 https://sw6-2012.googlecode.com/svn/branches/Group_3_Server/savannah local_name
\end{verbatim}

\part*{configuration}
Import the project into your favorite JAVA editor.
Savannah is currently hardwired to connect to a specific database, you will need to navigate to 
\begin{verbatim}
 dk.aau.cs.giraf.savannah.server.QueryHandler.java
\end{verbatim}
and change line 12 to the IP of your host machine and provide user name(currently eder), password(currently 123456) and database name(currently /04).

when done, export as a runnable jar and select
\begin{verbatim}
 ServerMain (dk.aau.cs.giraf.savannah/serverMain/ServerMain.java)
\end{verbatim}
as launch configuration.

\part*{.jar}
After exporting the jar, drop it somewhere on your server where your user has write access(~/server will do just fine)
open a terminal, if you have not already, cd to the directory and type
\begin{verbatim}
 java -jar savannah.jar
\end{verbatim}

\part*{Windows}
\label{windows}
If you wish to run savannah on a windows server, you will need to navigate	 to 
\begin{verbatim}
 dk.aau.cs.giraf.savannah.io.Configuration.java
\end{verbatim}
and change line 23 and 28 to something a windows machine understands.

\end{document}
