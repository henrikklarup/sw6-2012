\documentclass[a4paper,11pt,fleqn,twoside,openright,titlepage]{memoir} % Brug openright hvis chapters skal starte p hjresider; openany, oneside

%%%%%%%%%%%%%%% Qoutations %%%%%%%%%%%%
\newcolumntype{R}{>{\raggedleft\arraybackslash}X}%

\def\changemargin#1#2{\list{}{\rightmargin#2\leftmargin#1}\item[]}
\let\endchangemargin=\endlist
\newcommand{\myQuoteA}[2]{
\begin{changemargin}{.7in}{.7in}
\textbf{``}
#1
\textbf{''}
\ifthenelse{\isempty{#2}}%  If there is nothing stated here just print the shit. 
{}% if #1 true
{\vspace{-3mm}\begin{changemargin}{.9in}{.7in}\raggedleft{{\footnotesize--- #2}}\end{changemargin}}% if #1 false
\end{changemargin}
}

\newcommand{\myQuote}[1]{\myQuoteA{#1}{}}

%%%% FIXME macro %%%%
\newcommand{\todo}[1]{\fxnote{#1}}
\newcommand{\todoV}[1]{\fxfatal{#1}}
\newcommand{\todov}[1]{\fxfatal{#1}}

%%%% PACKAGES %%%%

% indsat af Magnus, Hvor skal denne placers? (verbatim)
\usepackage{verbatim}
\usepackage{casecontrol}
\usepackage{ifthen}
\usepackage{xifthen}

% Commands fra P4 indsat via andre filer
\input{functions/wordlist}

% Til regex %
\usepackage{algorithm}

%  Oversttelse og tegnstning  %
\usepackage[ansinew]{inputenc}					% Gr det muligt at bruge ,  og  i sine .tex-filer
\usepackage[english]{babel}							% Dansk sporg, f.eks. tabel, figur og kapitel
\usepackage[T1]{fontenc}								% Hjlper med orddeling ved ,  og . Stter fontene til at vre ps-fonte, i stedet for bmp					
\usepackage{latexsym}										% LaTeX symboler
\usepackage{xcolor,ragged2e,fix-cm}			% Justering af elementer
\usepackage{pdfpages}										% Gr det muligt at inkludere pdf-dokumenter med kommandoen \includepdf[pages={x-y}]{fil.pdf}	
\pretolerance=2500 											% Gr det muligt at justre afstanden med ord (hjt tal, mindre orddeling og mere space mellem ord)
\usepackage{ulem}                       % Gennemstregning af ord med koden \sout{}
\usepackage{fixltx2e}										% Retter forskellige bugs i LaTeX-kernen
\usepackage{listings}
																			
%  Figurer og tabeller  floats   %
\usepackage{flafter}										% Srger for at dine floats ikke optrder i teksten fr de er sat ind.
\usepackage{multirow}                		% Fletning af rkker
\usepackage{hhline}                   	% Dobbelte horisontale linier
\usepackage{multicol}         	        % Fletning af kolonner
\usepackage{colortbl} 									% Muligre farver i tabeller
%\usepackage{float}												% Gr det muligt at placere figurer hvor du vil.   \begin{figure}[!h] % Will not be floating.
\usepackage{wrapfig}										% Indsttelse af figurer omsvbt af tekst. \begin{wrapfigure}{Placering}{Strrelse}
\usepackage{graphicx} 									% Pakke til jpeg/png billeder
\pdfoptionpdfminorversion=6							% Muliggr inkludering af pdf dokumenter, af version 1.6 og hjere
	
%  Matematiske formler og maskinkode 
\usepackage{amsmath,amssymb,stmaryrd} 	% Bedre matematik og ekstra fonte
\usepackage{textcomp}                 	% Adgang til tekstsymboler
\usepackage{mathtools}									% Udvidelse af amsmath-pakken. 
\usepackage{eso-pic}										% Tilfj billedekommandoer p hver side
\usepackage{lipsum}											% Dummy text \lipsum[..]
\usepackage{rsphrase}										% Kemi-pakke til RS-stninger
\usepackage[version=3]{mhchem} 					% Kemi-pakke til lettere notation af formler

%  Referencer, bibtex og url'er  %
\usepackage{url}												% Til at stte urler op med. Virker sammen med hyperref
\usepackage[english]{varioref}						% Giver flere bedre mulighed for at lave krydshenvisninger
\usepackage{natbib}											% Litteraturliste med forfatter-r og nummerede referencer
\usepackage{xr}													% Referencer til eksternt dokument med \externaldocument{<NAVN>}

%  Floats  %
\let\newfloat\relax 										% Memoir har allerede defineret denne, men det gr float pakken ogs
\usepackage{float}

% ANDET SJOV %
\usepackage{lastpage}							% Automatisk tæl antal sider


\usepackage[footnote,draft,english,silent,nomargin]{fixme}		% Indst rettelser og lignende med \fixme{...} Med final i stedet for draft, udlses en error 																															for hver fixme, der ikke er slettet, nr rapporten bygges.

%%%% CUSTOM SETTINGS %%%%

%  Marginer  %
\setlrmarginsandblock{3.5cm}{2.5cm}{*}	% \setlrmarginsandblock{Indbinding}{Kant}{Ratio}
\setulmarginsandblock{2.5cm}{3.0cm}{*}	% \setulmarginsandblock{Top}{Bund}{Ratio}
\checkandfixthelayout 									% Laver forskellige beregninger og stter de almindelige lngder op til brug ikke memoir pakker

%	 Afsnitsformatering  %
\setlength{\parindent}{0mm}           	% Strrelse af indryk
\setlength{\parskip}{4mm}          			% Afstand mellem afsnit ved brug af double Enter
\linespread{1,1}												% Linie afstand

%  Litteraturlisten  %
\bibpunct[, ]{[}{]}{;}{n}{,}{,} 					% Definerer de 6 parametre ved Harvard henvisning (bl.a. parantestype og seperatortegn)
\bibliographystyle{plain}			% Udseende af litteraturlisten. Ligner dk-apali - mvh Klein

%  Indholdsfortegnelse  %
\usepackage{tocloft}

\makeatletter
\def\@part[#1]#2{%
    \ifnum \c@secnumdepth >-2\relax
      \refstepcounter{part}%
      \addcontentsline{toc}{part}{\protect\numberline{\thepart}#1}%NEW
    \else
      \addcontentsline{toc}{part}{#1}%
    \fi
    \markboth{}{}%
    {\centering
     \interlinepenalty \@M
     \normalfont
     \ifnum \c@secnumdepth >-2\relax
       \huge\bfseries \partname\nobreakspace\thepart
       \par
       \vskip 20\p@
     \fi
     \Huge \bfseries #2\par}%
    \@endpart}
\renewcommand*\l@part[2]{%
  \ifnum \c@tocdepth >\m@ne
    \addpenalty{-\@highpenalty}%
    \vskip 1.0em \@plus\p@
    \setlength\@tempdima{2.0em}%NEW: indentation for lines 2,3,... change according to your needs
    \begingroup
      \parindent \z@ \rightskip \@pnumwidth
      \parfillskip -\@pnumwidth
      \leavevmode\large\bfseries
      \advance\leftskip\@tempdima% NEW: comment out if no indentation required for lines 2,3,...
      \hskip -\leftskip
      #1\nobreak\hfil \nobreak\hb@xt@\@pnumwidth{\hss #2}\par
      \penalty\@highpenalty
    \endgroup
  \fi}
\makeatother

\setsecnumdepth{subsection}		 					% Dybden af nummerede overkrifter (part/chapter/section/subsection)
\maxsecnumdepth{subsection}							% ndring af dokumentklassens grnse for nummereringsdybde
\settocdepth{section} 								% Dybden af indholdsfortegnelsen


%  Visuelle referencer  %
\usepackage[colorlinks]{hyperref}			 	% Giver mulighed for at ens referencer bliver til klikbare hyperlinks. .. [colorlinks]{..}
\hypersetup{pdfborder = 0}							% Fjerner ramme omkring links i fx indholsfotegnelsen og ved kildehenvisninger 
\hypersetup{														%	Opstning af farvede hyperlinks
    colorlinks = false,
    linkcolor = black,
    anchorcolor = black,
    citecolor = black
}

\definecolor{gray}{gray}{0.80}					% Definerer farven gr

%Noget med PDF.
\usepackage{pdfpages} 


%  Opstning af figur- og tabeltekst  %
 	\captionnamefont{
 		\small\bfseries\itshape}						% Opstning af tekstdelen ("Figur" eller "Tabel")
  \captiontitlefont{\small}							% Opstning af nummerering
  \captiondelim{. }											% Seperator mellem nummerering og figurtekst
  \hangcaption													%	Venstrejusterer flere-liniers figurtekst under hinanden
  \captionwidth{\linewidth}							% Bredden af figurteksten
	\setlength{\belowcaptionskip}{10pt}		% Afstand under figurteksten
		
%  Navngivning  %
%\addto\captionsdanish{
%	\renewcommand\appendixname{Appendix}
%	\renewcommand\contentsname{Contents}	
%	\renewcommand\appendixpagename{Appendix}
%	\renewcommand\cftchaptername{\chaptername~}				% Skriver "Kapitel" foran kapitlerne i indholdsfortegnelsen
%	\renewcommand\cftappendixname{\appendixname~}			% Skriver "Bilag" foran bilagene i indholdsfortegnelsen
%	\renewcommand\appendixtocname{Appendix}
%}



%  Kapiteludssende  %
\definecolor{numbercolor}{gray}{0.7}			% Definerer en farve til brug til kapiteludseende
\newif\ifchapternonum

\makechapterstyle{jenor}{									% Definerer kapiteludseende -->
  \renewcommand\printchaptername{}
  \renewcommand\printchapternum{}
  \renewcommand\printchapternonum{\chapternonumtrue}
  \renewcommand\chaptitlefont{\fontfamily{pbk}\fontseries{db}\fontshape{n}\fontsize{25}{35}\selectfont\raggedleft}
  \renewcommand\chapnumfont{\fontfamily{pbk}\fontseries{m}\fontshape{n}\fontsize{1in}{0in}\selectfont\color{numbercolor}}
  \renewcommand\printchaptertitle[1]{%
    \noindent
    \ifchapternonum
    \begin{tabularx}{\textwidth}{X}
    {\let\\\newline\chaptitlefont ##1\par} 
    \end{tabularx}
    \par\vskip-2.5mm\hrule
    \else
    \begin{tabularx}{\textwidth}{Xl}
    {\parbox[b]{\linewidth}{\chaptitlefont ##1}} & \raisebox{-15pt}{\chapnumfont \thechapter}
    \end{tabularx}
    \par\vskip2mm\hrule
    \fi
  }
}																						% <--

\chapterstyle{jenor}												% Valg af kapiteludseende - dette kan udskiftes efter nske
\pagestyle{plain}																							% Valg af sidehoved og sidefod


%  Fjerner den vertikale afstand mellem listeopstillinger og punktopstillinger  %
\let\olditemize=\itemize							
\def\itemize{\olditemize\setlength{\itemsep}{-1ex}}
\let\oldenumerate=\enumerate						
\def\enumerate{\oldenumerate\setlength{\itemsep}{-1ex}}


%%%% CUSTOM COMMANDS %%%%

\newcommand{\method}[1]{\texttt{#1}}						%%Do change
\newcommand{\function}[1]{\method{#1}}				%%Do change
\newcommand{\fu}[1]{\function{#1}}						%%Don't change
\newcommand{\degree}{\ensuremath{^\circ}}

%  Billede hack  %
\newcommand{\figur}[4]{
		\begin{figure}[H] \centering
			\includegraphics[width=#1\textwidth]{billeder/#2}
			\caption{#3}\label{#4}
		\end{figure} 
}

%  Specielle tegn  %
\newcommand{\grader}{^{\circ}C}
\newcommand{\gr}{^{\circ}}
\newcommand{\g}{\cdot}

%  Promille-hack (\promille)  %
\newcommand{\promille}{%
  \relax\ifmmode\promillezeichen
        \else\leavevmode\(\mathsurround=0pt\promillezeichen\)\fi}
\newcommand{\promillezeichen}{%
  \kern-.05em%
  \raise.5ex\hbox{\the\scriptfont0 0}%
  \kern-.15em/\kern-.15em%
  \lower.25ex\hbox{\the\scriptfont0 00}}


\renewcommand{\emph}[1]{\textit{#1}}

%%%% ORDDELING %%%%
\hyphenation{hvad hvem hvor hvis MPa kon-se-kvens-klas-se bjl-ken byg-ning-er e-le-ment brud-li-ni-er frem-lbs-tem-pe-ra-tur-en Bjl-ker-ne ind-bls-nings-strm-ning-er-ne tem-pera-tur-gra-di-en-ten fjer-nes fjer-ner var-me-pum-pe-kreds-lb gen-nem Byg-nings-reg-le-men-tet Dan-vak Sys-tem-air kanal-sys-te-mer b-rings-af-stand-en}	


%%%%%%%%% List Environments %%%%%%%%%%%
\usepackage{listings}
\usepackage{color}

\definecolor{gray95}{gray}{.95}
\definecolor{gray92}{gray}{.92}
\definecolor{gray75}{gray}{.75}
\definecolor{gray45}{gray}{.45}

%\lstset{frame=shadowbox, rulesepcolor=\color{gray}, basicstyle=\footnotesize, numbers=left, numberstyle=\footnotesize, numbersep=5pt}
\lstdefinestyle{sourceC}
{ 
	numbers=left,
	numbersep=5pt, 
	stepnumber=1,
	captionpos=b,  %bottom
	keywordstyle=\color[rgb]{0,0,1},
	commentstyle=\color[rgb]{0.133,0.545,0.133},
	stringstyle=\color[rgb]{0.627,0.126,0.941},
	backgroundcolor=\color{gray95},
	frame=lrtb,
	framerule=0.5pt,
	linewidth=1.00\textwidth,
	tabsize=4,
	numberbychapter=true,
	basicstyle=\ttfamily\footnotesize,
	language=C,
	breaklines=true,
	showstringspaces=false,
	emph=[1]{endregion,region,get,set,enum,TASK},%%%%%%%%%%% Add new keywords here
	emph=[2]{TerminateTask,ecrobot_get_sonar_sensor,nxt_motor_set_speed,nxt_motor_get_count,systick_get_ms,display_clear,display_goto_xy,display_string,display_int,display_update,systick_wait_ms}, %%Functions specific to nxtOSEK and such.
	emphstyle=[1]{\color[rgb]{0,0,1}},
	emphstyle=[1]{\color[rgb]{0,0,1}},
	emphstyle=[2]{\color[rgb]{0.1,0.5,0.5}},
	float=htb,
	breakindent=20pt
}

\lstdefinestyle{sourceCode}
{ 
	numbers=left,
	numbersep=5pt, 
	stepnumber=1,
	captionpos=b,  %bottom
	keywordstyle=\color[rgb]{0,0,1},
	commentstyle=\color[rgb]{0.133,0.545,0.133},
	stringstyle=\color[rgb]{0.627,0.126,0.941},
	backgroundcolor=\color{gray95},
	frame=lrtb,
	framerule=0.5pt,
	linewidth=1.00\textwidth,
	tabsize=4,
	numberbychapter=true,
	basicstyle=\ttfamily\footnotesize,
	language=[Sharp]C,
	breaklines=true,
	showstringspaces=false,
	emph=[1]{value,endregion,region,get,set,then,terminateIf},%%%%%%%%%%% Add new keywords here
	emph=[2]{Tag,Problem,Person,List,NotSupportedException,TestMethod,ProblemSearch,Assert,
	EntityCollection,Department,IEnumerable,TimeSpan,DateTime},%%Classes
	emphstyle=[1]{\color[rgb]{0,0,1}},
	emphstyle=[2]{\color[rgb]{0.1,0.5,0.5}},
	float=htb,
	breakindent=20pt
}


\lstdefinelanguage{waldo}
{morekeywords={openMain,openFrame,terminateFrame,terminateMain,openFrameLoop,terminateFrameLoop,if,do,else,terminateIf,while,terminateWhile,function,
return,terminateFunction,void,rotate,collision,Agent,agent,addAgentToScene,removeAgentFromScene,moveForward,getSceneAgent,getSceneAgentlistLength,
getAgentX,getAgentY,getSceneWidth,getSceneHeight,setAgentSpectrum,setAgentX,setAgentY,canAgentSeeAgent,random,getAgentAngle,setAgentAngle},
sensitive=false,
morestring=[b]{"}
}

\lstdefinestyle{waldo}
{ 
	numbers=left,
	numbersep=5pt, 
	stepnumber=1,
	captionpos=b,  %bottom
	keywordstyle=\color[rgb]{0,0,1},
	commentstyle=\color[rgb]{0.133,0.545,0.133},
	stringstyle=\color[rgb]{0.627,0.126,0.941},
	backgroundcolor=\color{gray95},
	frame=lrtb,
	framerule=0.5pt,
	linewidth=1.00\textwidth,
	tabsize=4,
	numberbychapter=true,
	basicstyle=\ttfamily\footnotesize,
	language=waldo,
	breaklines=true,
	showstringspaces=false,
	emph=[1]{red,blue,green,yellow,black,white,darkGray,lightGray,grey,cyan,orange,pink,magenta,+,-,*,/,==,!,<=,>=,>,<,++,--,&&},%%Colours and Operators
	emph=[2]{int,string,double,bool,agent,Scene,Color,scene,color},
	%emphstyle=[1]{\color[rgb]{0,0,1}},
	emphstyle=[1]{\color[rgb]{0.6,0.2,0.2}},
	emphstyle=[2]{\color[rgb]{0.1,0.5,0.5}},
	float=htb,
	breakindent=20pt
}

\lstdefinestyle{grammar}
{ 
	numbers=left,
	numbersep=5pt, 
	stepnumber=1,
	captionpos=b,  %bottom
	keywordstyle=\color[rgb]{0,0,1},
	commentstyle=\color[rgb]{0.133,0.545,0.133},
	stringstyle=\color[rgb]{0.627,0.126,0.941},
	backgroundcolor=\color{gray95},
	frame=lrtb,
	framerule=0.5pt,
	linewidth=1.00\textwidth,
	tabsize=4,
	numberbychapter=true,
	basicstyle=\ttfamily\footnotesize,
	breaklines=true,
	showstringspaces=false,
	emph=[1]{value,endregion,region,get,set},%%%%%%%%%%% Add new keywords here
	emph=[2]{Tag,Problem,Person,List,NotSupportedException,TestMethod,ProblemSearch,Assert,
	EntityCollection,Department,IEnumerable,TimeSpan,DateTime},%%Classes
	emphstyle=[1]{\color[rgb]{0,0,1}},
	emphstyle=[2]{\color[rgb]{0.1,0.5,0.5}},
	float=htb,
	breakindent=20pt
}

\renewcommand{\lstlistingname}{Code snippet}%%Changing the caption to read ``Code snippet''
\renewcommand{\lstlistlistingname}{List of Code Snippets}
\lstset{escapeinside={(*}{*)}}%%Defines the escape and unescape chars


\newcommand{\HRule}{\rule{\linewidth}{0.5mm}}



\newcommand{\hiddensubsection}[1]{
\stepcounter{subsection}
%\subsection*{\arabic{section}.\arabic{subsection}\hspace{1em}{#1}}}
\subsection*{{#1}}}