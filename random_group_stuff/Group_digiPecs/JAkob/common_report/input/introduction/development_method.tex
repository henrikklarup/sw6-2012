\section{Development Method}

As mentioned earlier this is a learning project and we therefore have been required to use the same development method in our multi project groups. We have looked into two different methods \textit{XP}(eXtreme Programming) \cite{XP}, and Scrum \cite{SCRUM}, both are agile development methods.
As a part of our courses we have been further introduced to how each of the methods works and how they could be implemented in the development process. 

With the knowledge of both XP and Scrum, we in the multi project decided to use Scrum (Seen in Figure \vref{fig:ScrumFrameworkFlow}) along with Scrum of Scrums. 

\begin{figure}[ht]
	\centering
		\includegraphics[scale = 0.45]{images/ScrumFrameworkFlow.png}
	\caption{An Overview of the Scrum Framework. This image is from http://www.scrumalliance.org/system/resource\_files/0000/3827/Scrum\_Framework\_Flow\_3.png}
	\label{fig:ScrumFrameworkFlow}
\end{figure}

To enhance our use of Scrum, we customized elements of Scrum to fit our project. The changes in the customized version of Scrum, and Scrum of Scrums are:
\begin{itemize}
	\item The sprint length have been shortened to approximately 7 - 14 half days.
	\item Some degree of pair programming have been introduced.
	\item There is no project owner because this is a learning project.
	\item Everyone is attending the Scrum of Scrums meetings.
	\item The Scrum of Scrums meetings are only held once at sprint planning.
\end{itemize}

The benefits from choosing Scrum, and Scrum of Scrums with the customizations are that everyone, at all times, will be able to know what the vision of the project is, and how close every group is to achieving their individual goals of the vision.

As a part of Scrum we maintain close contact with the customers. This helps keep the product backlog up to date and correctly prioritized. The customers are presented with the vision of the project, as well as sprint demonstrations of the different incrementation of the product.