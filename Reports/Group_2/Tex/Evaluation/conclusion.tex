\chapter{Conclusion}
%\textit{Konklusionen paa projektet}
The problem definition in \autoref{sec:problem_def} states:

\begin{quotation}
How can we ease the daily life for children with ASD and their guardians, while complying with the study regulation? 
\end{quotation}

%keep abreast of = følge med situationene osv. eksempel: He tried to keep abreast of all developments (Han prøvede at følge all udviklinger)
To fulfill the study regulations the five project groups created one unified system.
The project groups have been using the same development method to ensure that all project groups have had the possibility to keep abreast of the multi project, \autoref{sec:dev_meth}.\\

Together did the groups develop the system GIRAF, a collection of the projects of the semester, of which this project has focused on the timer application, WOMBAT.
The timer application is designed to replace the physical timers that the guardians uses, \autoref{sec:sys_def} and \autoref{cha:design}.\\

The usability test proves that the application is user friendly for the educators in the test group, \autoref{sec:usability_results}.
Furthermore does the acceptance test prove that the children, understand the meaning of the timers and attached pictograms, \autoref{sec:accept_test}. 
The pictogram attachment is a new feature that the educators did not have available previously with the physical timers, see \autoref{InterviewMette}.\\

The timer application is configurable to match any timer the guardian use, \autoref{subsec:cust_arch}, further can the guardians attach pictograms or timers to the timers in progress as an which is an improvement compared to the physical timers, \autoref{subsubsec:attach}.\\ 

The timer application is easy to use, as proven by the tests, as well as improved in functionality compared to the physical timers, we therefore conclude that the timer application can ease the daily life for children with ASD and their guardians.\\


