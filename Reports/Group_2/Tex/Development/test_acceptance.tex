\section{Acceptance Test}

The purpose of an acceptance test\cite{misc:designInterSys} is to test if the system fits into the context it is designed for. In this case the main objectives is to find out if it feels natural for the educators to use the timer application instead of their regular hourglasses and time timers, and to find out if the children understand the digital version of the different timers.

To test for acceptability, the system needs to be tested in the context it is developed for, and therefore the contact person borrowed the tablet with the timer application for a few days, so that she could use the system in real life scenarios. 

\subsection{Results and Observations}

The contact person wrote a diary, to keep track of her experiences with the application. The contact person used the timer application herself, and she let some of the other educators try it as well. The diary can be found in appendix \ref{sec:acceptance}.

They wrote in the diary that the children understood the meaning of both the timers and the pictograms shown on the tablet. One of the children was more interested in watching the digital timer count down than in the activity he was meant to do, but beyond that there were no problems understanding the system.

Beside the diary, we can analyze on the timers they have used in the acceptance test by looking in the "Last Used" list (see appendix \ref{sec:acceptance}).

We can see in the list that they have used different timers with different "Done"-pictures and times, only one out of eight timers had a slightly different color. This could indicate that changing the colors on the timers is not very intuitive, or they do not need the functionality anyway.