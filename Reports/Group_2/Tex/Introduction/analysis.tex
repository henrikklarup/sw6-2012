\chapter{Analysis}
Through meetings with an educator at Birken, a kindergarten for children with ASD, we have learned about the importance of having access to well-designed communication tools when working with children with ASD.\\

At Birken they often use hourglasses and time timers, in different sizes and colors to visualize the progression of time to the children. The children can then associate the color and size of an hourglass with the time it represents.\\
Our contact person explained how the educators at the institution use pictograms to communicate with the children. The children have a schedule of the day, where all their daily activities are listed in pictograms, such that the children can always go to their schedules and find out what they are going to do next. Also activity instructions are listed with pictograms, i.e. in the bathroom there is a scheme showing what to do when going to the bathroom.\\
The pictograms used at Birken is from the software \textit{Boardmaker}\cite{web:boardmaker}. To use the pictograms from \textit{Boardmaker}, the educators have to print, cut, and laminate them.\\

The educators have to take timers and pictograms with them, everywhere they go. Since there is supposed to be more than one specific timespan available, the timers can take up much space. Furthermore the pictograms are lightweight, which can make it difficult to use them outside when it is bad weather or windy.