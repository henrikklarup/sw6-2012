\chapter{Analysis}
Through meetings with Mette Als Andreasen, an educator at Birken, a special kindergarten for children with autism, we have learned a lot about children with autism, and the importance of having access to well-designed communication tools.\\

At Birken they often use hourglasses, and other kinds of timers, in different sizes and colors to visualize the progression of time to the children. The children will then associate the color and size of an hourglass with the time it represents, and specific timers are always used when they are performing specific activities, i.e. they always spend 30 minutes on eating lunch.\\
Mette Als Andreasen also explained how they use pictograms to communicate with the children. They have a scheme for the day, where all their daily activities are listed in pictograms, so the children can always go to their schemes and see what they are going to do next. Also activity instructions are listed with pictograms, i.e. in the bathroom there is a scheme showing how to wash hands.\\
The pictograms used at Birken comes from a licensed piece of software called \textit{Boardmaker}\cite{web:boardmaker}. To use the pictograms, the educators have to choose and edite them on the computer, print them out, cut them out in small squares, and laminate them. After that they can be put to use either on the schemes or by showing them when needed.\\

In general the guardians need a lot of different tools all the time, and the tools are not very practical to transport. Therefore it would be practical to have a digital version of the timers and pictograms, so they would only have to bring one tablet, where all the needed tools are available. This leads to the system definition of this subproject, which is found in the next section.