\chapter*{Resume}
Oasis is an administration module for the GIRAF system. The GIRAF system is developed as part of a multi project. The multi project is a continuation of a previous multi project by a group of 6th semester Software students at Aalborg University.

Besides Oasis, other groups have developed modules for the GIRAF system. These includes; Launcher, WOMBAT, PARROT, and Savannah. Oasis have received requirements from different multi project groups. These requirements specifies how the Oasis module is structured.

Oasis consists of three parts; a database, a library, and a Oasis administration app. The database part of Oasis is called Oasis Local Db. The Oasis Local Db is used for storing the data, which is used by GIRAF applications. The library of Oasis is called Oasis Lib. The Oasis Lib is the library, which works as a connection between the Oasis Local Db and GIRAF applications. The administration application is called Oasis App. The Oasis App demonstrates some of the utilities the Oasis Lib offers.

To ensure the correctness of the Oasis Lib we enforced dynamic white box testing through unit tests. The unit tests are created in order to test the Oasis Lib and the Oasis Local Db. It is only the normal operation of the Oasis Lib and Oasis Local Db that has been tested. 
The usability of the Oasis App has been tested and the result of the test have been documented.

The Oasis tools are used in the development of various applications for the GIRAF platform. Although we released a fully functional version of Oasis, there still exists some requirements Oasis does not fulfill. These requirements gives the opportunity for further development.