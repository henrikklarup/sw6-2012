\chapter{Sprint Process}
\label{ch:SprintProcess}
In this chapter we describe the sprint process of our project. In the project period we underwent \textcolor{red}{HVOR MANGE?? KAN KUN FINDE 4 BURNDOWNS?} sprints. The length of the sprints was decided at every sprint meeting, but they were usually no longer than 10 half days long. At the end of every sprint period, we held a meeting to discuss what we learned through the sprint, and if all the tasks was finished. In the following a description of the different sprints is presented. The Burndown charts and sprint backlogs can be seen in the Appendix.

\section{Sprints}
\subsection{Sprint 1}
Sprint 1 started on date \textcolor{red}{??} and lasted 7 half days. In this sprint period we started making the local database, the model objects, that is supposed to wrap the data from the database, and some methods to the library, we got as requirements from the other groups.

\subsubsection{what we learned}
In this sprint we learned that we should be a bit better at estimating the time of each task and that we needed to break the task into bigger tasks, so each person would not need to take a new task every minute (\textcolor{red}{bedre forklaring!}.

\subsection{Sprint 2}
Sprint 2 started on date \textcolor{red}{??} and lasted 8 half days. In this sprint period we updated the local database and the library with the new requirements we got from the other groups. Besides that we used a little time on setting a counter-strike server up, so we had something to do socially between the groups.

\subsubsection{what we learned}
In this sprint we learned that we still should be a bit better at estimating the time of each task, because there was some tasks we did not even start on.

\subsection{Sprint 3}
Sprint 3 started on date \textcolor{red}{??} and lasted \textcolor{red}{??} half days. In this sprint period we continued on updating the local database and library with the new requirements gathered from the other groups. Besides that we started thinking on our "`BMI App"', which is not necessary a BMI application, but it is an application, which shows the functionality of the library and local database.

\subsubsection{what we learned}
In this sprint we learned that we should be a bit better at not taking new tasks in to the sprint backlog when the sprint has started, because this will prevent us from making any progress. Besides that we overestimated the size of the tasks, which lead to tasks not being completed before the end of the sprint, because they are connected to each other.

\subsection{Sprint 4}
Sprint 4 started on date \textcolor{red}{??} and lasted \textcolor{red}{??} half days. In this sprint period we continued on updating the local database and library with the new requirements gathered from the other groups. Besides that we started writing javadoc to the library, re factored the code in the local database, started making a method, which created dummy data, and wrote a section of the common report (Target Platform and Developing method).

\subsubsection{what we learned}
In this sprint we learned that we still needs to be better at saying no to other groups, which come with new tasks in the middle of the sprint period.