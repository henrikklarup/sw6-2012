Here we will conclude on our findings and discoveries throughout this project as well as what our experience of the different parts have been.

This semester we were introduced to a new way of developing software, through the use of an agile development method. 
The chosen method was Scrum and the tools new to us and typical for Scrum were: Project Backlog, Sprints, and Scrum meeting.

The project backlog was very useful for us, once we learned how to use it. Our problem was that we were too unspecific when defining assignments. Once we started thinking more about the assignments and make them more concrete we started benefiting more from the backlog because it provided a great overview of our progress.

The sprints was also a great success in the beginning of the project. We started out light because we did not had any idea of what we were able to accomplish in one sprint. After the first sprint our estimates became more precise.
Once we reached the fifth sprint we started to slack on the structure of sprints because we knew what was left to do and did not feel the needs to create sprint backlogs. This was also when we started to prepare the usability test and final release.
Once we decided that we were done programming we stopped using the sprint backlog altogether. We have not had a release of the system after every sprint but we do not feel that it has affected our result, because we have found alternate ways to test by utilizing usability analysis.

As mentioned earlier we did not utilize the sprint meeting because we did not feel the need to do so. We do not feel that we have lost anything by negating the meetings because we have had good communication in the group.

We think we have used the new work method to great success and we have been successful in neglecting tools we did not see the potential in using and implementing other tools that we needed. 
\subsubsection*{The database}
Our database design is functional and well tested. We have made a lot of small adjustments throughout the project based on requirements from the other groups. We decided to freeze the database after third sprint because we had to change it during every sprint so to be able to start implementing we felt it was necessary. We have worked closely with the Oasis group when designing the database since our database should be equivalent to theirs, and that turned out to be a great success.

\subsubsection*{The Web Interface}
Our web interface was quite a challenge because we had no experience in java servlet. This resultet in a slow start because we had to learn how to use it. This was not accounted in the sprints resulting in delay in the beginning. Once we started getting more confident with java servlet the process sped up and we started to finish features during sprints.
We have used a lot of redundant code in the classes since we are still inexperienced and this code should be refactored in some way. At the end of sprint 5 we decided that we were done programming. There are still features that has not been implemented, these are described in our design.
In Retrospect we should have chosen a smaller part of the system to implement so that we would have time to learn java servlet and refactored the code. We have been successful with testing of the web interface both in the form of a mock up test, test cases, and usability test.
We have not been able to correct the mistakes but have documented them for use by next year's project.

\subsubsection*{The Server Software}

insert stuff about the server software here