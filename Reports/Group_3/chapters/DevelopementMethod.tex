\begin{comment}
One of the goals for this semester is to show that we have knowledge in choosing and implementing different development methods, and apply the method that we find most suited for our project. In order to do this it is necessary to know the different methods, before choosing the best suited. 

Development methods can be put into two categories: Traditional and Agile.

Traditional methods emphasizes analysis and documentation. The structure of a traditional method is therefore often split into phases, for instance the analysis phase or the implementation phase. Once a phase is done you move to the next phase. An example of this is the waterfall method. %TODO [put some example]

Agile methods utilizes an iterative development style, where the focus is on getting a part of the system to work, and then as more iterations are completed, you add to the system until the system is ready to be released. An example of an agile development method
is Scrum. %TODO [put some example]

Both approaches have strengths and weaknesses. The traditional method features lots of documentation and analysis. It also creates a great overview of the whole system, and thus it is easier to estimate how long time it will take before the system is ready to ship. This makes traditional methods well suited for large projects. The agile method on the other hand implements the iteration driven approach. One of the great features of this approach is that you can correct errors and include things you might have forgotten relatively easy, because you just do it in the next iteration. 

One of the big drawbacks of the traditional method is that you cannot just go back a step, if you realize you have forgotten something or made an error. This is very costly because you have to do lots of steps over again. Another big issue with the traditional method is that the customers of the system might not always know what they want or need at the start of the development, but this is where you do all the analysis. The Agile method excels at this as you can present to your customers your current build and receive feedback that you might not have considered. 

The development method that we find best suited for our project is an agile method. This is because in this semester we are working together with educators from institutions working with children with ASD. They will have requirements for our system so it is important that we are taking them into the project. It is also important that we choose a development method that have tools for managing bigger project groups as we are working as part of a bigger team to develop a system.
\end{comment}

In \autoref{sec:devmeth} we decided on using Scrum of Scrums in the multi project.
This does not require us to use Scrum in the project group.
We have considered the following agile development methods:

\begin{itemize}
\item Scrum
\item XP
\end{itemize} 

In short Scrum is a method that emphasizes a self-directed and self-organizing team, each iteration is client driven meaning that the clients provides the requirements and features will be prioritized according to the clients needs\cite{larman}.

XP emphasizes programming and testing. This means that there are very little documentation of the system other than the code. A simple design is preferred and the code is refactored with high frequency. This method requires high discipline since most planning is done orally\cite{larman}.
