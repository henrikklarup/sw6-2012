\section{Unimplemented features}
\label{backlog_unimplemented}

\subsection{Child Mode}
\label{backlog:child_mode}
Another mode to implement in the launcher is called \emph{Child Mode}. 
This mode is only for \autists[] and is minded on them using it alone without a guardian. 
In this mode, all settings are removed and everytime an app is clicked, it launches directly, i.e. without profile selection, with the child's preferences instead of showing the profile selection screen.

\subsection{Custom Icons}
\label{backlog:custom_icons}
This feature can change the icon of an app in the app grid. 
This is important, as it allows \autists[] to use familiar icons, which was deemed important in \autoref{Preanalysis:Usability_for_children}. 
Recognition is important for children, and if they associate an app with another thing e.g. PARROT with a picture they know from their everyday, it should be possible for them to use the same in \giraf[].

\subsection{Add and Remove Apps}
\label{backlog:hide_apps}
This feature was meant to be integrated in the drawer. 
It should allow guardians to customize what apps are accessible in the home screen, both for them or for a child, when \nameref{backlog:child_mode} has been implemented. 

\subsection{Home Screen Modes}
\label{backlog:home_screen_modes}
The home screen should work in both portrait and landscape mode.

\subsection{Lock Screen}
\label{backlog:lock_screen}
The lock screen was a feature that WOMBAT needed for when a timer ran out and the tablet should not be usable any more. 
Currently, the authentication screen is available for this, but no functionality has been implemented to actually make it lock the device. 

\subsection{Logo Screen Loading}
\label{backlog:logo_screen_loading}
The logo screen is currently static, with no indication that the launcher is loading. 
This can make users unsure of whether they need to wait or do something else, something that might be fixed by implementing a loading animation on the logo screen.