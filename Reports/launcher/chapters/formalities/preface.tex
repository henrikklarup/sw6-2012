\pdfbookmark[1]{Preface}{Preface} % Bookmark name visible in a PDF viewer

\begingroup
\let\clearpage\relax
\let\cleardoublepage\relax
\let\cleardoublepage\relax

\section*{Preface} % Abstract name

\label{report_structure}

In deciding upon a report structure, two main approaches were considered.

\begin{enumerate}
	\item Traditional analysis, design, \& implementation-structured product oriented report.
	\item ``Diary'' iteration-structured and process oriented report.
\end{enumerate}

The strength of the first approach is the clear way the product would be presented, as it would be easy to understand the analysis, design, and implementation done of the product. 
The weakness is the ability to represent the reasons for the design choices and the implementational details, as the project has alternated between analysis, designing, and programming activities, as is customary in the iterative work process used. \newline
The strength of the second approach is that the ordering of events and decisions is chronological in the report, making the reasoning clearer and more intuitive to follow. 
The weakness is the lack of a clear and structured way to easily grasp and understand the final state of analysis, design, and implementational work done. \newline
We have chosen a mix of the two, where we first focus on process oriented activities, their chronological order, and what their effects were, followed by a product oriented description of analysis, design, and implementation.



% %\chapter{Preface}
% %\label{preface}
% 
% \paragraph{Quotations} are the words of another person along with a source.
% The source of the citation will either be in the text immediately before or after, or could potentially be incorporated into the quote as shown in the example below: \\
% 
% %\myQuoteA{This is an example of a quotation.}{X, p.~Y}
% 
% \paragraph{References} are references to sections, figures, code snippets, chapters or parts written elsewhere in the report. 
% This could look like the following:\\
% 
% This is explained in Section~X.Y.
% 
% \paragraph{Code examples} are written in a special environment so they are easy to read and recognize. 
% %Examples can be seen in Code~snippet~\ref{chelloworld}. 
% Whenever there is a sequence of three dots (``...'') in a code snippet, it means that we have omitted some content, which is not important in that specific context.
% 
% %\begin{lstlisting}[style=java, caption={Code example of a hello world program written in C.}, label=chelloworld]
% %#include<stdio.h>
% %
% %main()
% %{
% %    printf("Hello World");
% %}
% %\end{lstlisting}
% 
% 
% %Short summary of the contents\dots

\endgroup			

\vfill
