\pdfbookmark[1]{Preface}{Preface} % Bookmark name visible in a PDF viewer

\begingroup
\let\clearpage\relax
\let\cleardoublepage\relax
\let\cleardoublepage\relax

\chapter*{Preface} % Abstract name

\label{report_structure}

In deciding upon a report structure, two main approaches were considered.

\begin{enumerate}
	\item Traditional analysis, design, and implementation-structured product oriented report.
	\item ``Diary'' iteration-structured and process oriented report.
\end{enumerate}

The strength of the first approach is the clear way the product would be presented, as it may be more straightforward to understand the analysis, design, and implementation of the product. 
The weakness is the ability to represent the reasons for the design choices and the implementational details, as the project has alternated between analysis, designing, and programming activities, as is customary in the agile work process. 

The strength of the second approach is that the ordering of events and decisions is presented chronologically, making the reasoning clearer and more intuitive. 
The weakness is the lack of a clear and structured way to easily grasp and understand the final state of the analysis, design, and implementation.

A mix of the two have been chosen. 
We first focus on process oriented activities, their chronological order, and what their effects were. 
This is followed by a product oriented description of analysis, design, implementation, testing, validation, and lastly an epilogue. 

\endgroup			

\vfill
