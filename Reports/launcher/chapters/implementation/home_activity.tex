\section{HomeActivity}
The \activity{HomeActivity} implements the app management described in \autoref{sec:app_manangement}. 
As seen in \autoref{fig:appmanagement_design} in \nameref{sec:app_management}, there are three branches of interaction. 
One of these is the \emph{Launch app} functionality.
A prerequisite of launching an app, is that the app is available to the user. 
To find out which apps to load, two lists of apps are retrieved. 
The first is the apps attached to the user in the Oasis database, and the second is the apps installed on the device.
The apps to be loaded are computed by taking the intersection of these two lists.

This ensures users are not allowed access to apps on the device they are restricted from.

The launcher does not ensure that apps are installed. 
A consequence hereof is that even if a user has permission to access an app, it will not be accessible if not installed.

\subsection{Screenshots}




\subsection{Drawer}
The drawer is implemented as described in \autoref{sec:drawer}.


\begin{figure}[h!]
	\centering
	\includegraphics[scale=0.2]{gfx/home-activity_closed}
	\caption{Shows the launcher with the drawer closed}
	\label{fig:home-activity_closed}
\end{figure}

\begin{figure}[h!]
	\centering
	\includegraphics[scale=0.2]{gfx/home-activity_open}
	\caption{Shows the launcher with the drawer opened. The color picker can also be seen here.}
	\label{fig:home-activity}
\end{figure}

\subsection{Color picker}
\label{home:colorpicker}
To show the functionality of the drawer a color picker for apps were implemented and hidden in the drawer. The color picker can be seen in \autoref{fig:home-activity} as mentioned in \autoref{backlog:homebar_drawer} and explained in \autoref{design:app_manangement}.
The \textit{color picker} is a tool that was made to change colors on apps, this have been extended so far that if the logo of e.g. WOMBAT is choosen to be red then the background in WOMBAT will also be set to red when the application is loaded.

The color picker is implemented with a color board with ten predefined \giraf[] colors. These colors can be assigned to any app that are in the launcher and there is no limit on how many times the user can use one color. This means that if the user want they can make every app e.g. red.
The color picker is made with drag and drop functionality as described in \autoref{par:colorpicker}. When the user have choosen a color and assigned it the app icon will change to this color and the color is saved in the local database so that if the user restarts the launcher the apps will still have the colors the users have assigned for them. 
The color picker can be seen in \autoref{fig:home-activity}

\subsection{Widgets}
The widgets which is described in \autoref{par:widgets} and can be seen in \autoref{fig:home-activity}is between the app board and the drawer, this is called the \textit{home bar}. Here it is also possible to see a picture of the user curretly logged in. The user can logout with the orange button in the button of the home bar. There are two widgets: One that shows the connectivity to the Savanah server. This widget has three states: Connected and updated, connected and synchronising and not connected. Connected and updated tells the user that all data is synchronized with the server and they are good to go. Connected and synchronising means that either the tablet needs data from the server or to push some changes to the server. Not connected tells the user that they are not connected to the server and therefore can not update or synchronize with the server.
If the user clicks on the connectivity widget they will get a description to the currently showing status telling them what this icon means.
The second widget is a calendar widget which tells the user what day it is in the month in a numeric format. If the user clicks this widget they will get a description with day, d. dd, month, week. This is done so the user easily can access and see what day it is.


\begin{lstlisting}[style=sourceCode, language=JAVA, caption=This is code, label=lst:homeActivity] 
\end{lstlisting}