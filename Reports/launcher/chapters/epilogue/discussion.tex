\chapter{Discussion}
This section will discuss the decisions made through out the project, and their relevance and ramifications. 
\todo{Check for multi-project vs. multi project}
The management of the multi-project was conducted by the \localgroup[]s, and no project manager was chosen.
Having no leader role in the project meant that excessive time was spent making decision as a majority of the \globalgroup[] had to be convinced of the quality of an proposal.
In this scenario a leader might have been able to cut down the time spent discussing, and force through a decision.
A possible drawback of having a leader, could be that he or she overrules too many discussions and diminishes the involvement of the \globalgroup[] as whole.

In this semester the \localgroup[]s have a hard limit member count of four \citep{web:rammestudyreg}, instead of the previous six.
The lowered amount of group members demands more effort from each member.
Absence also affects the group in a greater way, as group size lowers.

The backlog is a crucial element in agile development and it is used to encourage developers and help communicating across the multi-project.
Backlogs were created in each \localgroup[], but there were not created a global one, for the \globalgroup[].
As seen in \autoref{iterative:sprint1}, there were confusion with what tasks each \localgroup[] were handling.
An effort to develop a global backlog might have helped unite and create a clearer target for the \globalgroup[], and helped promote cooperation.
For example, Savannah might have prioritized synchronization higher.
This might have eased the process of testing, as each group would not have to create their own dummy data for their local Oasis database.

Using XP, a goal is to motivate the team members by letting them choose their own tasks, so they can pick what they feel motivated for.  This can however create some difficulties, when no one in the team feels compelled to do a certain task.
Working in pairs to solve non-programming tasks might have solved the issue, as each member in the pair help motivate each other. 

Communication the process of each \localgroup[] were done by using burn down charts, amongst other things.
The burn down charts were hard to get accurate readings from, as each \localgroup[] had their own measurement of time.
It might not be needed to have burn down charts for further development. 

The essence of working iterative is the fact your previous work might change.
To accommodate this we decided to write our report after implementation was done. This reduced overhead of keeping the link between the implementation and the report up-to-date.
However, writing after the implementation creates a sort of traditional working process.
Traditional processes have issues if something goes out of schedule, and this issue can also be problematic in this case.

	% - Project manager? !
	% - Group size (Sickness ! 
	% - Global feature prioritizing / Vision vs. Feature !
	% - Isolation? (Savannah) !
	% - Task distribution? / Why pair is important and not solo !
	% - Burndown charts? / Needed? !
	% - Writing after iteration? !
	% 
