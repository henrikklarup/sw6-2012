\section{Remarks to the next group of developers}
\label{epi:remarks}
This section is made for the group that comes after us and gets to develop on the launcher. Here it will be possible to find information on what future work we would have done but also which information and experiences we made working in an multi-project group and using agile development.

\subsection{Iterative process}
\label{epi:iterative_process}

When working in an iterative process it is important too meet often in the start of the project periode and later on in the project periode it is important not too have too many meetings. The reason for this is that in the early stages it is really important that all groups have the same vision and know what all the others are doing so dependencies etc. can be figured out and backlogs can be made! Later on it is important that the groups have time to work with their own project, but they should have open door policy, so they get the job done and do not waste their time with meetings which does benefit anything.
We found out that in the startup and onto the rapport writing it was great with sprint with a length of a week or two where often there would be sprint meetings every monday and friday.
We also figured that having people incharge was very important because then the job gets done and everyone knows who will take care of a subject and how to talk to if they have problems with this subject.


\subsection{Development}
\label{epi:development}

The forward development of the launcher is very important because it provides functionality both to the users of the \giraf[] system but also all other apps in \giraf[].

\subsection{Modes}
\label{epi:modes}
The two modes represented in the \giraf[] system is the guardian- and child mode. As seen in \autoref{chap:backlog} only the guardian mode was implemented. Therefore it is possible that the next step the launcher should take is to include child mode so children can use the tablet alone without an guardian.

\subsection{\giraf[] GUI components}
\label{epi:giraf_GUI_components}

