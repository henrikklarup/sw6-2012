\section{Understanding}
\label{interviews}

We set the scope of this project to include the \guardian[]s as both customers and users, who can make demands of the product, while the \autists[] are considered users. 
The needs of the \autists[] are conveyed by the \guardian[]s.
For this reason, the \guardian[]s are the customers of the \giraf[] launcher, and are at the same time expert users of the customer domain. 
To gain good understanding of the customer domain, each of the customers available in the project, were interviewed. 
The interviews were semi structured, in order to create a solid base of questions, while having flexibility, in case exploring additional topics would become relevant. \citep[p. 152]{dieb-book}

The results from the interviews showed that usability and flexibility are critical for the customers. 
Usability was requested based on experience with previously used solutions, including expensive learning courses, and flexibility was deemed critical, due to individual \autists[] potentially having different needs.

\subsection{Field Observations}
As part of the interviews were conducted at \egebakken{}\footnote{An institution for \autists[] and the workplace of our customer.}, observations were also made. 
These observations gave insight into the tools used (physical and digital), the robustness of the environment and the organization needed for the \autists[]. 

\subsubsection{Impressions}
A concern could be that the physical environment was unsuitable for the use of tablets, however we found nothing to be concerned about.

The physical tools used at \egebakken{} fulfill simple tasks, and are robust, but also come with limitations. 
While they can be very flexible, they can require a sizable effort to work with. 
Some of these tools are suitable for having digital replacements, as this could lessen the previous issue. 
Software solutions do however already exist, but our customers find them hard to use and overly expensive. 
These solutions also come with limited flexibility, limiting the work of the \guardian[]s, as they try to optimize the solution to best enhance the understanding and communication skills of each \autist[]. 
The \guardian[]s are also in charge of organizing the daily schedule of each \autist[], with the child having some limited influence. 