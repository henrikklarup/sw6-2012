\section{Usability Considerations}
\label{Preanalysis:Usability_for_children}
This section describes general usability concerns that are relevant for the product. \newline

The \giraf[] system is designed to be usable by both children and adults, making the age range of potential users very large, as it can be used by young children all the way up to elderly \guardian[]s. 
This is amplified by the fact that the children using the system have ASD, which can result in their mental capacity being below that associated with their physical age. \newline
While \autists[] are not directly customers for the product this semester, it is expected that the product will be expanded to accomodate them later on. 
For this reason, considerations like "`The product needs to accommodate children who click madly around a screen as well as those who sit back and wait to be told what to do."' \citep{microsoft:usability} and "`The results also provide further evidence that young children require interactions designed specifically for their developing motor skills."' \citep[p. 8]{mousesize} are relevant. 
Another aspect that is important is consistency, as children "`... are also better at recognition over recalling"' \citep{microsoft:usability}, but concistency generally increases usability, regardless of age \citep[page 90]{dieb-book}. 
This includes creating concistency with the real world, using familiar icons and behavior in the product \citep{microsoft:usability}. 