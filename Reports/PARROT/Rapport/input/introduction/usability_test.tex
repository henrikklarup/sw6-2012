\section{Usability Test}
\label{common:sec:usability_test}
As stated in the motivation, quality assurance through testing of the system is required. Therefore a usability test was conducted in order to measure to current usability of the GIRAF platform as a whole, as well as of the individual parts of the platform. Furthermore, the next wave of developers will immediately be able to start correcting the found usability issues.

\subsection{Approach}
The test group for the test is the five contact persons. We assess that they, as a test group, are representative. We base this on them being a mix of pedagogues and teachers, with varying computer skills.

Even if they have some knowledge about the overall idea of the GIRAF platform, and although some of the contact persons had previously informally used some aspects or parts of the system, they had not been exposed to the platform as a whole, and therefore still are of value.

The invitation sent to the test persons can be found in \autoref{appendice:usability_test}.\\

Based on the fact that the test should be short and the test group is small, the Instant Data Analysis (IDA) method for usability was chosen. \cite{usability:ida}

A traditional video analysis method could have been used, but was not estimated to be time-effective enough to be chosen.

\subsubsection*{Setup}
The usability test is divided into two tests: a test of the three applications, and a test of the two administration applications -- tablet application and web application.
Each test is assigned a team to accommodate the need to run two tests simultaneously.
The teams are made with respect to the criteria of the Instant Data Analysis process.\\
Each team consisted of:

\begin{itemize}
	\item 1 x Test Coordinator
	\item 1 x Test Monitor
	\item 1 x Data Logger
	\item 2 x Observers
\end{itemize}

The usability lab on Aalborg University is designed with two rooms for usability testing and a control room to observe and record the tests.
The two test chambers were assigned a test each and the control room were used to observe both tests as seen in figure \ref{fig:test_setup}.

\begin{figure}[H]
	\centering
		\includegraphics[width=\textwidth]{input/images/test_setup.png}
	\caption{An overview of the usability lab at Cassiopeia, Department of Computer Science, Aalborg University.}
	\label{fig:test_setup}
\end{figure}

As an precaution, all tests were recorded on video and audio.

\subsubsection*{Execution}

\begin{figure}[H]
	\centering
		\includegraphics[width=\textwidth]{input/images/usability_testschedule.png}
	\caption{The schedule of the usability test.}
	\label{fig:usability_testschedule}
\end{figure}

The tests were conducted according to the schedule in \autoref{fig:usability_testschedule}.

Briefing, debriefing, and questionnaire documents can be found in \autoref{app:usability_documents}, and the results of the test can be found in \autoref{sec:usability_results}.