\section{XML Layouts} %%Name of Feature.
\subsection*{Problem}
We need to establish the layout of our application, so that the user can navigate and manipulate it.
\subsection*{Solution}
The Android SDK utilizes, in part, an XML based method of creating the visual layouts or views that is seen in the application.
GridView, LinearLayout, and ImageView are all examples of Layout Elements used in the XML documents.
Eclipse with the Android Development Tools installed also allows for a drag and drop approach to setting up the XML document, by moving elements on a visual representation of it.

\subsection*{Execution}
The following is an example of XML code and an portion of the ``speechboard_layout''
\begin{source}{}
<?xml version="1.0" encoding="utf-8"?>
<LinearLayout xmlns:android="http://schemas.android.com/apk/res/android"
    android:layout_width="match_parent"
    android:layout_height="125dp"
    android:orientation="horizontal">

    <ImageView
        android:id="@+id/catpic"
        android:layout_width="100dp"
        android:layout_height="100dp"
        android:layout_marginLeft="6dp" />

    <TextView
        android:id="@+id/catname"
        android:layout_width="wrap_content"
        android:layout_height="fill_parent"
        android:layout_marginLeft="10dp"
        android:gravity="center_vertical"
        android:textColor="#FF000000" />

</LinearLayout>
\end{source}


\subsection*{Result}

\subsection*{Notes}
The names of XML documents cannot contain capital letters.

\subsection*{Further Reading}
